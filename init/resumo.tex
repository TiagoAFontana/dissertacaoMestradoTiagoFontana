\textoResumo{ 

As ferramentas de Physical Design devem lidar com uma grande quantidades de dados para resolver problemas de circuitos com milhões de células.
Tradicionalmente, as ferramentas de \textit{Eletronic Design Automation} (EDA) são implementadas usando o paradigma de programação orientado a objetos (OOD).
No entanto, usar esse paradigma pode levar a objetos excessivamente complexos que resultam em desperdício de espaço de memória.
Esse desperdício de memória prejudica a exploração da localidade da memória cache e, consequentemente, degrada o tempo de execução do software.
Este trabalho propõe uma organização eficiente dos dados para três algoritmos diferentes de \textit{Physical Design}.
Para esta organização de dados, este trabalho aplica o modelo orientado a dados \textit{Data-Oriented Design}(DOD).
Diferentemente do modelo tradicional orientado a objetos, o modelo de programação de orientado a dados se concentra em como os dados são organizados na memória.
Como consequência, este modelo de programação pode explorar melhor a localidade espacial da cache e reduzir o tempo total de execução.
Para avaliar o impacto do uso do modelo de programação Data-Oriented Design, implementamos dois protótipos de software (uma implementação sequencial e uma paralela) de algoritmos de \textit{Physical Design} para cada um dos modelo de programação.
Os resultados experimentais preliminares mostraram que a implementação do orientada a dados (Data-Oriented Design) reduz o número total de \textit{cache misses} e é mais rápido do que a implementação com o paradigma Orientado a Objetos.

}
\palavrasChave{ 
Localidade da Cache; Eletronic Design Automation; Physical Design; Otimização de Software; Data-Oriented Design.
}

\textAbstract{ 

Physical design tools must handle huge amounts of data in order to solve problems for circuits with millions of cells. 
Traditionally, Electronic Design Automation (EDA) tools are implemented using Object-Oriented Design (OOD).
However, using this paradigm may lead to overly complex objects that result in waste of cache memory space.
This memory wasting harms cache locality exploration and, consequently, degrades software runtime. 
This work proposes an efficient organization of the data for three different algorithms of Physical Design.
For this data organizatio, this work apply Data-Oriented Design.
Differently from the traditional Object-Oriented design, the Data-Oriented Design programming model focus on how the data is organized in the memory.
As consequence, this programming model may better explore cache spatial locality and reduce the total runtime.
In order to evaluate the impact of using the Data-Oriented Design programming model, we implemented two software prototypes (a sequential and a parallel implementation) of Physical Design algorithms for each programming model.
Preliminary experimental results showed that the Data-Oriented Design implementation reduces the total number of cache misses and it is more faster than Object-Oriented Design implementation.

}
\keywords{ 
Cache Locality; Eletronic Design Automation; Physical Design; Software Optimization; Data-Oriented Design.
}