\textoResumo{ 

\textit{Game engines} contemporâneas devem lidar eficientemente com enormes quantidades de dados para renderizar gráficos 3D em imagens de altíssima resolução, modelar sistemas físicos realistas e também processar sistemas complexos de inteligência artificial.
Para atender a esses requisitos, vários conceitos e padrões de projetos são aplicados durante o desenvolvimento de um jogo para explorar as arquiteturas modernas de computadores.
Análogo a estas, as ferramentas de Síntese Física  devem lidar com uma grande quantidades de dados para resolver problemas de circuitos com milhões de células.
% Tradicionalmente, estas ferramentas são implementadas seguindo o modelo de programação orientado a objetos (OOD).
Para representar os componentes pertencentes a síntese física pode-se modela-los utilizando o modelo de programação orientado a objetos (OOD).
No entanto, usar esse modelo pode levar a objetos excessivamente complexos que resultam em desperdício de espaço de memória.
Esse desperdício de memória prejudica a exploração da localidade espacial pela memória \textit{cache} e, consequentemente, degrada o tempo de execução do \textit{software}.
Este trabalho propõe uma organização eficiente dos dados para diferentes etapas da Síntese Física (\textit{Physical Design}).
Para esta organização de dados, este trabalho aplica o modelo orientado a dados \textit{Data-Oriented Design} (DOD).
Diferentemente do modelo tradicional orientado a objetos, o modelo de programação orientado a dados se concentra em como os dados são organizados na memória.
Como consequência, este modelo de programação proporciona uma melhoria na localidade espacial do programa possibilitando assim que mais dados úteis sejam recuperados para a memória cache, e assim, reduzir o tempo total de execução.
% Para avaliar o impacto do uso do modelo de programação DOD, foram implementados dois protótipos de software (uma implementação sequencial e uma paralela) de algoritmos da Síntese Física (\textit{Physical Design}) para cada um dos modelo de programação.
Para avaliar o impacto da organização dos dados nas etapas pertencentes à Síntese Física, este trabalho compara o número de cache misses e o tempo de execução para protótipos de \textit{software} entre os modelo DOD e OOD com execuções sequenciais e paralelas.
Os resultados experimentais mostraram que a implementação com o modelo DOD reduziu em até cinco ordens de grandeza o número total de \textit{cache misses} no melhor cenário e executou tão rápido quanto o modelo OOD no cenário mais desfavorável.
}
\palavrasChave{ 
Localidade da Cache; Electronic Design Automation; Síntese Física; Otimização de Software; Data-Oriented Design.
}

\textAbstract{ 


Modern game engines must efficiently handle huge amounts of data to render 3D graphics for very-high resolution images, model realistic physical systems, and also process complex artificial intelligence systems. 
To fulfill such requirements, several concepts and design patterns are applied during the game development to take advantage of modern computer architectures.
Similary, Physical Design tools must handle huge amounts of data in order to solve problems for circuits with millions of cells. 
To represent the components belonging to the Physical Design can be modeled using Object-Oriented Design (OOD).
However, using this paradigm may lead to overly complex objects that result in the waste of cache memory space.
This memory wasting harms the exploitation of locality by the cache memory and, consequently, degrades software runtime. 
This work proposes an efficient organization of the data for different Physical Design phases.
The data organization evaluated in this work is called Data-Oriented Design (DOD).
Differently from the traditional Object-Oriented Design, the Data-Oriented Design programming model focus on how the data is organized in the memory.
As a consequence, this programming model may better explore cache spatial locality and reduce the total runtime.
In order to evaluate the impact of the data organization on different Physical Design phases, this work compares the number of cache misses and runtime for software prototypes between the DOD and OOD models with sequential and parallel executions.
The experimental results showed that DOD implementations can reduce the number of cache misses up to five orders of magnitude in the best scenario and executed as fast as OOD model in the most unfavorable scenario.


}
\keywords{ 
Cache Locality; Electronic Design Automation; Physical Design; Software Optimization; Data-Oriented Design.
}

% texto sem comandos latex

% Título: 
% Avaliação quantitativa do impacto da organização dos dados na execução de programas: estudos de caso no contexto da Síntese Física

% Resumo:
% Game engines contemporâneas devem lidar eficientemente com enormes quantidades de dados para renderizar gráficos 3D em imagens de altíssima resolução, modelar sistemas físicos realistas e também processar sistemas complexos de inteligência artificial. Para atender a esses requisitos, vários conceitos e padrões de projetos são aplicados durante o desenvolvimento de um jogo para explorar as arquiteturas modernas de computadores. Análogo a estas, as ferramentas de Síntese Física  devem lidar com uma grande quantidades de dados para resolver problemas de circuitos com milhões de células. Para representar os componentes pertencentes a síntese física pode-se modela-los utilizando o modelo de programação orientado a objetos (OOD). No entanto, usar esse modelo pode levar a objetos excessivamente complexos que resultam em desperdício de espaço de memória. Esse desperdício de memória prejudica a exploração da localidade espacial pela memória cache e, consequentemente, degrada o tempo de execução do software. Este trabalho propõe uma organização eficiente dos dados para diferentes etapas da Síntese Física (Physical Design). Para esta organização de dados, este trabalho aplica o modelo orientado a dados Data-Oriented Design (DOD). Diferentemente do modelo tradicional orientado a objetos, o modelo de programação orientado a dados se concentra em como os dados são organizados na memória. Como consequência, este modelo de programação proporciona uma melhoria na localidade espacial do programa possibilitando assim que mais dados úteis sejam recuperados para a memória cache, e assim, reduzir o tempo total de execução. Para avaliar o impacto da organização dos dados nas etapas pertencentes à Síntese Física, este trabalho compara o número de cache misses e o tempo de execução para protótipos de software entre os modelo DOD e OOD com execuções sequenciais e paralelas. Os resultados experimentais mostraram que a implementação com o modelo DOD reduziu em até cinco ordens de grandeza o número total de cache misses no melhor cenário e executou tão rápido quanto o modelo OOD no cenário mais desfavorável.

% Abstract
% Modern game engines must efficiently handle huge amounts of data to render 3D graphics for very-high resolution images, model realistic physical systems, and also process complex artificial intelligence systems. To fulfill such requirements, several concepts and design patterns are applied during the game development to take advantage of modern computer architectures. Similary, Physical Design tools must handle huge amounts of data in order to solve problems for circuits with millions of cells. To represent the components belonging to the Physical Design can be modeled using Object-Oriented Design (OOD). However, using this paradigm may lead to overly complex objects that result in the waste of cache memory space. This memory wasting harms the exploitation of locality by the cache memory and, consequently, degrades software runtime. This work proposes an efficient organization of the data for different Physical Design phases. The data organization evaluated in this work is called Data-Oriented Design (DOD). Differently from the traditional Object-Oriented Design, the Data-Oriented Design programming model focus on how the data is organized in the memory. As a consequence, this programming model may better explore cache spatial locality and reduce the total runtime. In order to evaluate the impact of the data organization on different Physical Design phases, this work compares the number of cache misses and runtime for software prototypes between the DOD and OOD models with sequential and parallel executions. The experimental results showed that DOD implementations can reduce the number of cache misses up to five orders of magnitude in the best scenario and executed as fast as OOD model in the most unfavorable scenario.