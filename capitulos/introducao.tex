\chapter{Introdução}
\label{cap:introducao}
    % evolução da tecnologia -> transistores menores
    % circuitos modernos muito grande grandes -> ferramentas de EDA
    % eletrônica de consumo -> tempo limitado para projeto de um circuito
    % é benéfico a redução de tempo na execução dos algoritmos sem perda da qualidade da solução
    % memória representa gargalo -> explorar a localidade na cache
    %
    % uma possível solução que não altera a qualidade da solução é o melhor armazenamento dos dados

A evolução da tecnologia de fabricação de \acp{ic} vem permitindo uma redução drástica e contínua das dimensões dos transistores.
O consequente aumento de densidade permite a fabricação de \acp{ic} com um número cada vez maior de transistores, podendo chegar a milhões de transistores.
Além disso, os circuitos contemporâneos devem respeitar uma ampla gama de restrições, sejam elas de atraso, potência e\@/\@ou área.
Adicionalmente, o processo de projeto e fabricação possui um tempo limitado para que um novo eletronico possa garantir o mercado.
Portanto, as Ferramentas de \ac{eda} são obrigatórias no design de \acp{ic} modernos.




ferramentas seguem um fluxo de projeto
explicar o fluxo
varias etapas do fluxo podem voltar
dados das etapas podem ser reutilizados
é benefico uma melhor organização dos dados


Essas ferramentas de \ac{eda} devem tratar um grande volume de dados em um curto período de tempo. Portanto, essas ferramentas de \ac{eda} devem empregar otimizações de software, como o uso de melhores estruturas de dados, paralelização e exploração de local de cache.

modelagem dos dados deve ser ótima para diversas relações
OOD pode gerar uma hierarquia muito complexa e problemas sem solução (a não ser replicação do código)








este trabalho avalia o impacto da exploração da localidade do cache no agrupamento de registros.


\section{Justificativa}
    o que a literatura não fez
    Portanto, é desejável o desenvolvimento ... sobretudo uma comparação que faça uso de uma infraestrutura realista.

\section{Objetivos e Contribuições Pretendidas}
        Objetivo
            Este trabalho tem como objetivo a avaliação quantitativa do impacto de diferentes
        Objetivos específicos

        Contribuições científicas e tecnológicas
            avaliação quantitativamente do desempenho causado pela modelagem dos dados em três tarefas da síntese física
            Resultados experimentais utilizando uma infraestrutura baseada em circuitos industriais,

\section{Metodologia}
        Implementar versões de problemas
        avaliar quantitativamente o número de cache misses
        avaliar quantitativamente o tempo de execução

\section{Limitações(escopo) deste trabalho}
        somente 3 problemas avaliados
        somente uma arquitetura de processador e cache
        paralelização com um chunk único

\section{Organização deste trabalho}
