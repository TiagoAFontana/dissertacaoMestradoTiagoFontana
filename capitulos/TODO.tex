\chapter{Introdução}
    evolução da tecnologia -> transistores menores
    circuitos modernos muito grande grandes -> ferramentas de EDA 
    eletrônica de consumo -> tempo limitado para projeto de um circuito
    é benéfico a redução de tempo na execução dos algoritmos sem perda da qualidade da solução 
    memória representa gargalo -> explorar a localidade na cache
    
    uma possível solução que não altera a qualidade da solução é o melhor armazenamento dos dados

    \section{Justificativa}
    o que a literatura não fez
    Portanto, é desejável o desenvolvimento ... sobretudo uma comparação que faça uso de uma infraestrutura realista.
    
    \section{Objetivos e Contribuições Pretendidas}
        Objetivo
            Este trabalho tem como objetivo a avaliação quantitativa do impacto de diferentes
        Objetivos específicos
        
        Contribuições científicas e tecnológicas
            avaliação quantitativamente do desempenho causado pela modelagem dos dados em três tarefas da síntese física
            Resultados experimentais utilizando uma infraestrutura baseada em circuitos industriais,

    \section{Metodologia}
        Implementar versões de problemas
        avaliar quantitativamente o número de cache misses
        avaliar quantitativamente o tempo de execução
        
    \section{Limitações(escopo) deste trabalho}
        somente 3 problemas avaliados
        somente uma arquitetura de processador e cache
        paralelização com um chunk único

\chapter{Trabalhos correlatos}
    trabalhos que exploram a localidade dos dados
    trabalhos de limitação do OOD?
    trabalhos que falam DOD?
    % Trabalhos que otimizem o K-means

\chapter{Técnica Proposta}
    apresentar modelo de memoria
    apresentar como cache miss funciona
    apresentar ood
    apresentar limitação do ood
    apresentar dod
    apresentar entity system
    Discutir como o DOD reduz o numero de cache misses
    como aplicar o dod em problemas
        Problema A - limites do chip
            como seria modelado OOD
            como seria modelado DOD
        Problema B - interconexão
            como seria modelado OOD
            como seria modelado DOD
        Problema C - cluster
            como seria modelado OOD
            como seria modelado DOD