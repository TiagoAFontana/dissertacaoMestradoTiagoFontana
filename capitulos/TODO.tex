\chapter{Introdução}
    evolução da tecnologia -> transistores menores
    circuitos modernos muito grande grandes -> ferramentas de EDA 
    eletrônica de consumo -> tempo limitado para projeto de um circuito
    é benéfico a redução de tempo na execução dos algoritmos sem perda da qualidade da solução 
    memória representa gargalo -> explorar a localidade na cache
    
    uma possível solução que não altera a qualidade da solução é o melhor armazenamento dos dados

    \section{Justificativa}
    o que a literatura não fez
    Portanto, é desejável o desenvolvimento ... sobretudo uma comparação que faça uso de uma infraestrutura realista.
    
    \section{Objetivos e Contribuições Pretendidas}
        Objetivo
            Este trabalho tem como objetivo a avaliação quantitativa do impacto de diferentes
        Objetivos específicos
        
        Contribuições científicas e tecnológicas
            avaliação quantitativamente do desempenho causado pela modelagem dos dados em três tarefas da síntese física
            Resultados experimentais utilizando uma infraestrutura baseada em circuitos industriais,


falar que ood é imprescindível no mundo atual de software
however
ood não organiza adequadamente os dados


\chapter{Conceitos Fundamentais}


\chapter{Trabalhos correlatos}
    trabalhos que exploram a localidade dos dados
    trabalhos de limitação do OOD?
    trabalhos que falam DOD?
    % Trabalhos que otimizem o K-means

\chapter{EXPLORANDO A LOCALIDADE DOS DADOS EM PHYSICAL DESIGN AUTOMATION}
    apresentar modelo de memoria
    apresentar como cache miss funciona
    apresentar ood
    apresentar limitação do ood
    apresentar dod
    apresentar entity system
    Discutir como o DOD reduz o numero de cache misses
    
\section{caracterização de algoritmos de physical design}
    este capitulo visa caracterizar e identificar quais algoritmos de physical design podem se beneficiar do uso do modelo de programação orientado a dados

    \section{escolha dos algoritmos}      
        apresentar o conjunto de características que foi identificado na seção anterior
        discutir quais dessas características são benéficas no modelo DOD e quais destas não se aplica ao modelo DOD (ILP)
        apresentar quais problemas este trabalho irá avaliar

    \section{DOD em algoritmos de Physical Design}
    como aplicar o dod em problemas
        Problema A - limites do chip
            como seria modelado OOD
            como seria modelado DOD
        Problema B - interconexão
            como seria modelado OOD
            como seria modelado DOD
        Problema C - cluster
            como seria modelado OOD
            como seria modelado DOD
        Problema D - roteamento das nets
            como seria modelado OOD
            como seria modelado DOD


\chapter{Resultados Experimentais}
    \section{infraestrutura e configuração experimental}
        \subsection{arquitetura zeus}
        \subsection{implementação e compilação}
            implementação em C++
            gcc 5.4.0
            flags: -O3 -ftree-vectorize -fopenmp
            
        \subsection{escolha de ferramenta para medições}
            PAPI (http://icl.cs.utk.edu/papi/)
                executa na máquina hospedeira
                somente arquitetura da maquina hospedeira
                utiliza hardware counters
                possibilita medr L1, L2 e L3 (instruções e dados)
                
            Valgrind - cachegrind (http://valgrind.org/)
                simula arquitetura (simulates how your program interacts with a machine's cache hierarch)
                possibilita diversas arquiteturas
                somente medições na L1 e Last Level
                
            Perf (https://www.gnu.org/software/gperf/)
                executa na máquina hospedeira
                somente arquitetura da maquina hospedeira
                utiliza hardware counters
                somente medições na L1 e Last Level
                
            Intel VTune (https://software.intel.com/en-us/intel-vtune-amplifier-xe)
                pago
                precisa compilar o código com ICC
    
    \section{resultados problema A}
    \section{resultados problema B}
    \section{resultados problema C}
    \section{resultados problema D}

\chapter{Conclusões e Trabalhos Futuros}




    \section{Limitações(escopo) deste trabalho}
        número de problemas limitados
        sem análise de programação dinâmica
        somente uma arquitetura de processador e cache
        paralelização com um chunk único