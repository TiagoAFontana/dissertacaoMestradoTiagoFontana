Data e Horário:
06 de Julho de 2018
14:00 hs

Orientador:
José Luís Almada Güntzel
E-mail:
j.guntzel@ufsc.br

DADOS DO TRABALHO:
Título: 
Avaliação quantitativa do impacto da organização dos dados na execução de programas: estudos de caso no contexto da Síntese Física

Resumo:
Game engines contemporâneas devem lidar eficientemente com enormes quantidades de dados para renderizar gráficos 3D em imagens de altíssima resolução, modelar sistemas físicos realistas e também processar sistemas complexos de inteligência artificial. Para atender a esses requisitos, vários conceitos e padrões de projetos são aplicados durante o desenvolvimento de um jogo para explorar as arquiteturas modernas de computadores. Análogo a estas, as ferramentas de Síntese Física  devem lidar com uma grande quantidades de dados para resolver problemas de circuitos com milhões de células. Para representar os componentes pertencentes a síntese física pode-se modela-los utilizando o modelo de programação orientado a objetos (OOD). No entanto, usar esse modelo pode levar a objetos excessivamente complexos que resultam em desperdício de espaço de memória. Esse desperdício de memória prejudica a exploração da localidade espacial pela memória cache e, consequentemente, degrada o tempo de execução do software. Este trabalho propõe uma organização eficiente dos dados para diferentes etapas da Síntese Física (Physical Design). Para esta organização de dados, este trabalho aplica o modelo orientado a dados Data-Oriented Design (DOD). Diferentemente do modelo tradicional orientado a objetos, o modelo de programação orientado a dados se concentra em como os dados são organizados na memória. Como consequência, este modelo de programação proporciona uma melhoria na localidade espacial do programa possibilitando assim que mais dados úteis sejam recuperados para a memória cache, e assim, reduzir o tempo total de execução. Para avaliar o impacto da organização dos dados nas etapas pertencentes à Síntese Física, este trabalho compara o número de cache misses e o tempo de execução para protótipos de software entre os modelo DOD e OOD com execuções sequenciais e paralelas. Os resultados experimentais mostraram que a implementação com o modelo DOD reduziu em até cinco ordens de grandeza o número total de cache misses no melhor cenário e executou tão rápido quanto o modelo OOD no cenário mais desfavorável.

Palavras-chave:
Localidade da Cache; Electronic Design Automation; Síntese Física; Otimização de Software; Data-Oriented Design.

-------
PRODUÇÕES QUE CUMPREM OS REQUISITOS ESTABELECIDOS PARA AGENDAMENTO DA DEFESA:
ISPD
- TÍTULO DO ARTIGO: How Game Engines Can Inspire EDA Tools Development: A use case for an open-source physical design library
- AUTORES: Tiago Augusto Fontana; Renan Netto; Vinicius Livramento; Chrystian Guth; Sheiny Almeida; Laércio Pilla; José Luís Güntzel
- VEÍCULO DE PUBLICAÇÃO: Proceedings of the 2017 ACM on International Symposium on Physical Design (ISPD 2017)
- CLASSIFICAÇÃO DO VEÍCULO NO QUALIS CC: B1
- DOI NUMBER: http://dx.doi.org/10.1145/3036669.3038248

SBCCI
- TÍTULO DO ARTIGO: Exploiting Cache Locality to Speedup Register Clustering
- AUTORES: Tiago Augusto Fontana; Sheiny Almeida; Renan Netto; Vinicius Livramento; Chrystian Guth;  Laércio Pilla; José Luís Güntzel
- VEÍCULO DE PUBLICAÇÃO: Proceedings of the 30th Symposium on Integrated Circuits and Systems Design (SBCCI 2017)
- CLASSIFICAÇÃO DO VEÍCULO NO QUALIS CC: B2
- DOI NUMBER: http://dx.doi.org/10.1145/3109984.3110005


# INDICAÇÃO DA BANCA EXAMINADORA
# ATENÇÃO!
# - DOS DOCENTES EXTERNOS AO PPGCC, informar: nome, instituição de origem, participação por videoconferência, e-mail e link do currículo lattes.
# - DOS DOCENTES CREDENCIADOS NO PPGCC, informar: nome e e-mail

PRESIDENTE: 
nome: José Luís Almada Güntzel
instituição de origem: Universidade Federal de Santa Catarina - UFSC
e-mail: j.guntzel@ufsc.br
currículo lattes: http://lattes.cnpq.br/3431795837830476
Bolsista de Produtividade em Pesquisa 2

MEMBRO (1):
nome: Fernando Gehm Moraes
instituição de origem: Pontifícia Universidade Católica do Rio Grande do Sul - PUCRS
participação por videoconferência: Sim
e-mail: fernando.moraes@pucrs.br
currículo lattes: http://lattes.cnpq.br/2509301929350826
Bolsista de Produtividade em Pesquisa 1C 

MEMBRO (2):
nome: Cristina Meinhardt
instituição de origem: Universidade Federal do Rio Grande - FURG
participação por videoconferência: Não
e-mail: cris.meinhardt@gmail.com
currículo lattes: http://lattes.cnpq.br/6816603089210442

MEMBRO (3):
nome: Márcio Bastos Castro
instituição de origem: Universidade Federal de Santa Catarina - UFSC
e-mail: marcio.castro@ufsc.br
currículo lattes: http://lattes.cnpq.br/6876016315737507
Bolsista de Produtividade em Pesquisa 2

# MEMBROS SUPLENTES:
MEMBRO (1):
nome: Omar Omar Paranaiba Vilela Neto
instituição de origem: Universidade Federal de Minas Gerais - UFMG
participação por videoconferência: Sim
e-mail: omar@dcc.ufmg.br
currículo lattes: http://lattes.cnpq.br/6799776599317117
Bolsista de Produtividade em Pesquisa 2

MEMBRO (2):
nome: Luiz Cláudio Villar dos Santos
instituição de origem: Universidade Federal de Santa Catarina - UFSC
e-mail: luiz.santos@ufsc.br 
currículo lattes: http://lattes.cnpq.br/7115792628126490


COMENTÁRIOS (se houver):

