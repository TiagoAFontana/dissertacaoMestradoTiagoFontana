% !TEX root = ../dissertacao.tex
\chapter{Lista de Publicações e Prêmios}
\label{ap:producoes}
\section{Artigos publicados diretamente relacionados ao tema de mestrado}

A avaliação quantitativa da melhoria na organização dos dados, resultou em duas publicações diretamente relacionadas ao tema deste trabalho de mestrado. Os detalhes destas publicações estão listados nesta Seção.

\subsection{\textit{Proceedings of the 2017 ACM on International Symposium on Physical Design.}, 2017}

A Avaliação do impacto causado pelo número de caches misses em duas tarefas da síntese física de circuitos integrados, resultou na publicação de um atrigo completo no ISPD2017 --- International Symposium on Physical Design.

\begin{itemize}
\item \textbf{Qualis CC 2016:} B1
\item \textbf{Título:} \textit{How Game Engines Can Inspire EDA Tools Development: A use case for an open-source physical design library}
\item \textbf{Autores:}  \textbf{Tiago Augusto Fontana}; Renan Netto; Vinicius Livramento; Chrystian Guth; Sheiny Almeida; Laércio Pilla; José Luís Güntzel
\item \textbf{DOI:} http://dx.doi.org/10.1145/3036669.3038248
\item \textbf{\textit{Abstract:}} \emph{Similarly to game engines, physical design tools must handle huge amounts of data. Although the game industry has been employing modern software development concepts such as data-oriented design, most physical design tools still relies on object-oriented design. Differently from object-oriented design, data-oriented design focuses on how data is organized in memory and can be used to solve typical object-oriented design problems. However, its adoption is not trivial because most software developers are used to think about objects' relationships rather than data organization. The entity-component design pattern can be used as an efficient alternative. It consists in decomposing a problem into a set of entities and their components (properties). This paper discusses the main data-oriented design concepts, how they improve software quality and how they can be used in the context of physical design problems. In order to evaluate this programming model, we implemented an entity-component system using the open-source library Ophidian. Experimental results for two physical design tasks show that data-oriented design is much faster than object-oriented design for problems with good data locality, while been only sightly slower for other kinds of problems.}
\end{itemize}


\subsection{\textit{Proceedings of the 30th Symposium on Integrated Circuits and Systems Design}, 2017}

A avaliação do impacto na exploração da localidade da cache utilizando Data-Oriented Design (DOD) no contexto de clusterização de registradores gerou o trabalho denominado "Exploiting Cache Locality to Speedup Register Clustering", o qual foi apresentado oralmente e publicado nos anais do SBCCI2017 --- 30th Symposium on Integrated Circuits and Systems Design.

\begin{itemize}
\item \textbf{Qualis CC 2016:} B2
\item \textbf{Título:} \textit{Exploiting Cache Locality to Speedup Register Clustering}
\item \textbf{Autores:}  \textbf{Tiago Augusto Fontana}; Sheiny Almeida; Renan Netto; Vinicius Livramento; Chrystian Guth;  Laércio Pilla; José Luís Güntzel
\item \textbf{DOI:} http://dx.doi.org/10.1145/3109984.3110005
\item \textbf{\textit{Abstract:}} \emph{Physical design tools must handle huge amounts of data in order to solve problems for circuits with millions of cells. Traditionally, Electronic Design Automation tools are implemented using Object-Oriented Design. However, using this paradigm may lead to overly complex objects that result in waste of cache memory space. This memory wasting harms cache locality exploration and, consequently, degrades software runtime. This work proposes applying Data-Oriented Design on the register clustering problem. Differently from the traditional Object-Oriented design, the Data-Oriented Design programming model focus on how the data is organized in the memory. As consequence, this programming model may better explore cache spatial locality. In order to evaluate the impact of using the Data-Oriented Design programming model for register clustering, we implemented two software prototypes (a sequential and a parallel implementation) of the K-means clustering algorithm for each programming model. Experimental results showed that the sequential Data-Oriented Design implementation is on average $7.5\%$ faster when compared to the Object-Oriented Design implementation, while its parallel version is $15\%$ faster when compared to the Object-Oriented one.}
\end{itemize}

\section{Prêmios}
\subsection{2017 CAD Contest (Problem C: Multi-Deck Standard Cell Legalization) @ ICCAD 2017}

A modelagem dos dados proposta nesta dissertação foi utilizada para suportar a técnica de legalização submetida à competição ICCAD 2017 CAD Contest (problem C: Multi-Deck Standard Cell Legalization), a qual proporcionou à equipe do Embedded Computing Laboratory da UFSC ficar entre os três primeiros colocados.
O resultado oficial será dibulgado na cerimônia de entrega de prêmios do ICCAD 2017,que realizar-se-á em 13 de novembro de 2017, na cidade de Irvine, Califórnia, EUA.

\begin{itemize}
\item \textbf{Equipe:} Ophidian (\texttt{cada001})
\item \textbf{Membros:} Renan Netto, Tiago Augusto Fontana, Sheiny Fabre, Thiago Barbatto, Chrystian Guth, Prof. José Güntzel e Prof. Laercio Lima Pilla
\item \textbf{Colocação:} $3^{\circ}$ Lugar
\end{itemize}
