% !TEX root = ../dissertacao.tex
% \acresetall{}

\chapter{Conclusões e Trabalhos Futuros}
\label{cap:conclusao_e_trabalhos_futuros}

% revisar um pouco sobre programação
    
% falar da organização proposta (entity component system)
% falar dos benefícios da organização (resultados)
%     falar do ordenamento que melhorou o caso 2
%       vetorizou o código
%       facilitou a paralelização

% trabalhos futuros   
%     analisar um estudo de caso com programação dinâmica (abacus).

Este trabalho caracterizou as etapas da síntese física e elegeu um subconjunto destas como representantes das principais características da organização dos dados em tais etapas.
Além disso, o trabalho também investigou a utilização do modelo de \ac{dod} para este subconjunto de etapas, realizando comparações com o modelo \ac{ood}.
Para auxiliar no acesso aos dados modelados com \ac{dod}, propôs-se o uso do padrão de projeto \textit{Entity-Component System}, o qual foi estendido para suportar o agrupamento de propriedades que possuam relações de composição e agregação.

A avaliação quantitativa do número de \textit{cache misses} comprovou que o modelo \ac{dod} explorou a localidade espacial fornecida pela memória cache, reduzindo o número de \textit{cache misses} significativamente na maioria dos casos estudados.
Esta redução foi muito expressiva (obtendo até $99\%$ de redução) para tarefas em que poucos atributos/propriedades são necessários e a ordem do acesso a estes não é relevante, como por exemplo no estudo de caso 1.
No cenário onde mais atributos/propriedades são necessários sem importar a ordem de acesso (como o descrito pelo estudo de caso 3) a redução no número de  \textit{cache misses} foi insignificativa no cenário sequencial.
Contudo, no mesmo cenário, com execução paralela a redução no número de \textit{cache misses} atingiu cerca de $30\%$.
Já no cenário com mais atributos/propriedades e no qual a ordem do acesso aos dados é relevante (exemplificado pelo estudo de caso 2), o modelo \ac{dod} com agrupamento de propriedades atingiu uma redução de $85\%$ na execução sequencial e $90\%$ na execução paralela .
Para tarefas que possuem mapeamento sobre grafos (como o estudo de caso 4), a redução no número de  \textit{cache misses} foi de $52\%$ na execução sequencial.

Estas reduções no número de  \textit{cache misses} impactaram no tempo total de execução das tarefas.
O modelo \ac{dod} foi pelo menos tão rápido quanto o modelo \ac{ood}, sendo que executou mais rápido na maioria dos estudos de caso.
Este modelo também permitiu que o compilador gerasse um número maior de instruções vetoriais (\ac{simd}).
Assim, este trabalho constatou que modelar os dados relativos à síntese física pode reduzir o tempo total necessário para o desenvolvimento de um \ac{ic}.

Como trabalhos futuros é preciso avaliar algoritmos que utilizem programação dinâmica.
Isso garantiria a completude na representatividade das características presentes na síntese física.
Um algoritmo possível para esta análise é o Abacus~\cite{spindler2008abacus} pertencente a etapa de legalização.
Outro trabalho futuro seria comprovar que algoritmos de outros domínios, com as mesmas características elencadas por este trabalho, possuem comportamento semelhante ao reportado pelos resultados experimentais apresentados no presente trabalho. Com isso, seria possível gerar uma ontologia com as classes de algoritmos e determinar o possível desempenho segundo suas características.
