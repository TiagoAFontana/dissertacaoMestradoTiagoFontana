% !TEX root = ../dissertacao.tex
% \acresetall{}

\chapter{Conclusões e Trabalhos Futuros}
\label{cap:conclusao_e_trabalhos_futuros}

% revisar um pouco sobre programação
    
% falar da organização proposta (entity component system)
% falar dos benefícios da organização (resultados)
%     falar do ordenamento que melhorou o caso 2
%       vetorizou o código
%       facilitou a paralelização

% trabalhos futuros   
%     analisar um estudo de caso com programação dinâmica (abacus).

Este trabalho caracterizou as etapas da síntese física e elegeu um subconjunto destas para representarem as principais características da organização dos dados/métodos.
Este trabalho propôs a utilização do modelo de \ac{dod} para este subconjunto e realizou comparações com o modelo tradicional \ac{ood}.
Para auxiliar no acesso aos dados modelados com \ac{dod} foi utilizado o padrão de projeto \textit{Entity-Component System}.
Este padrão foi estendido para suportar o agrupamento de propriedades que possuam relações de composição e agregação.

A avaliação quantitativa do número de \textit{cache misses}, para cada estudo de caso, mostrou que o modelo \ac{dod} explorou a localidade espacial fornecida pela memória cache, reduzindo este número significativamente.
Esta redução foi muito expressiva (obtendo até $99\%$ de redução) para tarefas em que poucos atributos/propriedades são necessários e a ordem do acesso a estes não é relevante, como por exemplo no estudo de caso 1.
No cenário em que mais atributos/propriedades são necessários sem importar a ordem de acesso, estudo de caso 3, a redução no número de cache misses foi insignificativa no cenário sequencial.
Contudo, no mesmo cenário, com execução paralela         a redução atingiu cerca de $30\%$.
Já no cenário com mais atributos/propriedades e no qual a ordem do acesso aos dados é relevante, o modelo \ac{dod}, com agrupamento de propriedades, atingiu uma redução de $85\%$ na execução sequencial e $90\%$ na execução paralela (estudo de caso 2).
Para o estudo de caso 4, representando tarefas sobre grafos, a redução no número de cache misses foi de $52\%$ na execução sequencial.

Estas reduções no número de cache misses impactaram no tempo total de execução das tarefas.
O modelo \ac{dod} foi pelo menos tão rápido quanto o modelo \ac{ood}, sendo que executou mais rápido na maioria dos estudos de caso.
Este modelo permitiu também que ao compilador gerar um número maior de instruções vetoriais.
Assim, este trabalho constatou que modelar os dados pertencentes a síntese física pode reduzir o tempo total necessário para o desenvolvimento de um \ac{ic}.

Como trabalhos futuros é preciso avaliar algoritmos que utilizem programação dinâmica.
Isso garantiria uma maior completude na representatividade das características presentes na síntese física.
Um algoritmo possível para esta análise é o Abacus~\cite{spindler2008abacus} pertencente a etapa de legalização.
