 \chapter{Técnica Proposta}
\label{cap:tecnica_proposta}

% Este capítulo apresenta a técnica proposta para resolver o problema de legalização incremental formulado na Seção \ref{sec:formulacao_incremental}, a qual adapta o algoritmo de \citeonline{chow2014cell} para utilizar uma estrutura de dados especializada no armazenamento de dados geométricos chamada R-tree. Inicialmente, este capítulo descreve, por meio de um exemplo, o processo de legalização incremental. Em seguida, apresenta como a R-tree realiza operações espaciais de forma rápida, para enfim apresentar os detalhes da adaptação proposta.

% \section{Processo de legalização incremental}

apresentar modelo de memoria
apresentar como cache miss funciona
apresentar ood
apresentar limitação do ood
apresentar dod
apresentar entity system
Discutir como o DOD reduz o numero de cache misses
como aplicar o dod em problemas
    Problema A - limites do chip
        como seria modelado OOD
        como seria modelado DOD
    Problema B - interconexão
        como seria modelado OOD
        como seria modelado DOD
    Problema C - cluster
        como seria modelado OOD
        como seria modelado DOD