% !TEX root = ../dissertacao.tex
\chapter{Cronograma}
\label{cap:cronograma}

Este capítulo apresenta as atividades já concluídas até o momento da escrita deste documento, assim como as atividades pendentes que serão realizadas após o exame de qualificação.

\section{Atividades concluídas}

\begin{itemize}
    \item \textbf{C1}: Implementar problemas de \textit{Physical Design} utilizando \ac{ood} e \ac{dod};
    \item \textbf{C2}: Avaliar quantitativamente os resultados obtidos na Atividade C1;
    \item \textbf{C3}: Escrita e submissão do artigo para o ISPD 2017 (Qualis  CC 2016: B1);
    \item \textbf{C4}: Implementar técnica de clusterização de elementos utilizando \ac{ood} e \ac{dod};
    \item \textbf{C5}: Avaliar quantitativamente os resultados obtidos na Atividade C4;
    \item \textbf{C6}: Escrita e submissão do artigo para o SBCCI 2017 (Qualis  CC 2016: B2);
    \item \textbf{C7}: Apresentação do artigo no SBCCI 2017;
    \item \textbf{C8}: Escrita do texto do Exame de Qualificação.
    % \item \textbf{C9}:
    % \item \textbf{C10}:
\end{itemize}

\section{Atividades pendentes}

\begin{itemize}
    \item \textbf{P1}: Realizar as sugestões de melhoria do trabalho propostas pela banca no exame de qualificação;
    \item \textbf{P2}: Implementar Sistema de Entidade com suporte a acesso ordenado das propriedades de uma entidade;
    \item \textbf{P3}: Adaptar problemas já implementados nas atividades C1 e C4 para ordenarem as propriedades pela ordem de acesso;
    \item \textbf{P4}: Avaliar quantitativamente os resultados obtidos na Atividade P3;
    \item \textbf{P5}: Escrever e submeter artigo para revista ou evento de índice restrito com as contribuições totais deste trabalho;
    \item \textbf{P6}: Escrever a dissertação de mestrado;
    \item \textbf{P7}: Preparar a apresentação e defender este trabalho de mestrado;
    \item \textbf{P8}: Realizar as correções sugeridas pela banca durante a defesa;
    \item \textbf{P9}: Entregar a versão final da dissertação de mestrado na Biblioteca Central da Universidade Federal de Santa Catarina.
\end{itemize}

\section{Cronograma}

A Figura~\ref{tab:cronograma} apresenta o cronograma o previsto para realização deste trabalho de mestrado.
Os meses de Novembro e Dezembro serão dedicados às correções sugeridas pela banca no exame de qualificação, assim como a implementação do ordenamento no sistema de entidades.
Finalizando as correções, entre Janeiro e Março pretende-se terminar toda a análise experimental o que suportará a escrita de um artigo, que será realizado nos meses de Março e Abril.
Por fim, os meses entre Maio e Julho são reservados para realizar as correções sugeridas pela banca na defesa e entrega do artigo na Biblioteca Central da Universidade Federal de Santa Catarina.
Desta forma, a finalização deste trabalho de mestrado está dentro do prazo de 24 meses previsto pelo programa de Pós-Graduação em Ciência da Computação da Universidade Federal de Santa Catarina.


\begin{table}[b]
\centering
\caption[Cronograma do trabalho]{Cronograma previsto para realização do trabalho de mestrado}
\label{tab:cronograma}
\resizebox{\textwidth}{!}{
\begin{tabular}{@{}cccccccccc@{}}
\toprule
                                  & \textbf{Novembro}      & \textbf{Dezembro}      & \textbf{Janeiro}       & \textbf{Fevereiro}     & \textbf{Março}         & \textbf{Abril}         & \textbf{Maio}          & \textbf{Junho}         & \textbf{Julho}         \\ \midrule
\multicolumn{1}{|c|}{\textbf{P1}} & \multicolumn{1}{c|}{X} & \multicolumn{1}{c|}{}  & \multicolumn{1}{c|}{}  & \multicolumn{1}{c|}{}  & \multicolumn{1}{c|}{}  & \multicolumn{1}{c|}{}  & \multicolumn{1}{c|}{}  & \multicolumn{1}{c|}{}  & \multicolumn{1}{c|}{}  \\ \midrule
\multicolumn{1}{|c|}{\textbf{P2}} & \multicolumn{1}{c|}{}  & \multicolumn{1}{c|}{X} & \multicolumn{1}{c|}{X} & \multicolumn{1}{c|}{}  & \multicolumn{1}{c|}{}  & \multicolumn{1}{c|}{}  & \multicolumn{1}{c|}{}  & \multicolumn{1}{c|}{}  & \multicolumn{1}{c|}{}  \\ \midrule
\multicolumn{1}{|c|}{\textbf{P3}} & \multicolumn{1}{c|}{}  & \multicolumn{1}{c|}{X} & \multicolumn{1}{c|}{X} & \multicolumn{1}{c|}{X} & \multicolumn{1}{c|}{}  & \multicolumn{1}{c|}{}  & \multicolumn{1}{c|}{}  & \multicolumn{1}{c|}{}  & \multicolumn{1}{c|}{}  \\ \midrule
\multicolumn{1}{|c|}{\textbf{P4}} & \multicolumn{1}{c|}{}  & \multicolumn{1}{c|}{X} & \multicolumn{1}{c|}{X} & \multicolumn{1}{c|}{X} & \multicolumn{1}{c|}{}  & \multicolumn{1}{c|}{}  & \multicolumn{1}{c|}{}  & \multicolumn{1}{c|}{}  & \multicolumn{1}{c|}{}  \\ \midrule
\multicolumn{1}{|c|}{\textbf{P5}} & \multicolumn{1}{c|}{}  & \multicolumn{1}{c|}{}  & \multicolumn{1}{c|}{}  & \multicolumn{1}{c|}{}  & \multicolumn{1}{c|}{X} & \multicolumn{1}{c|}{X} & \multicolumn{1}{c|}{}  & \multicolumn{1}{c|}{}  & \multicolumn{1}{c|}{}  \\ \midrule
\multicolumn{1}{|c|}{\textbf{P6}} & \multicolumn{1}{c|}{X} & \multicolumn{1}{c|}{X} & \multicolumn{1}{c|}{X} & \multicolumn{1}{c|}{X} & \multicolumn{1}{c|}{X} & \multicolumn{1}{c|}{X} & \multicolumn{1}{c|}{X} & \multicolumn{1}{c|}{}  & \multicolumn{1}{c|}{}  \\ \midrule
\multicolumn{1}{|c|}{\textbf{P7}} & \multicolumn{1}{c|}{}  & \multicolumn{1}{c|}{}  & \multicolumn{1}{c|}{}  & \multicolumn{1}{c|}{}  & \multicolumn{1}{c|}{}  & \multicolumn{1}{c|}{}  & \multicolumn{1}{c|}{X} & \multicolumn{1}{c|}{X} & \multicolumn{1}{c|}{}  \\ \midrule
\multicolumn{1}{|c|}{\textbf{P8}} & \multicolumn{1}{c|}{}  & \multicolumn{1}{c|}{}  & \multicolumn{1}{c|}{}  & \multicolumn{1}{c|}{}  & \multicolumn{1}{c|}{}  & \multicolumn{1}{c|}{}  & \multicolumn{1}{c|}{}  & \multicolumn{1}{c|}{X} & \multicolumn{1}{c|}{X} \\ \midrule
\multicolumn{1}{|c|}{\textbf{P9}} & \multicolumn{1}{c|}{}  & \multicolumn{1}{c|}{}  & \multicolumn{1}{c|}{}  & \multicolumn{1}{c|}{}  & \multicolumn{1}{c|}{}  & \multicolumn{1}{c|}{}  & \multicolumn{1}{c|}{}  & \multicolumn{1}{c|}{}  & \multicolumn{1}{c|}{X} \\ \bottomrule
\end{tabular}
}
\end{table}
%
